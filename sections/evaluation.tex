\chapter{Evaluation}

\section{Examples of Tweets and Errors}
\begin{itemize}
	\item Explain a bit about the source of the data, size etc.
        \begin{itemize}
          \item Tweets come from 1TB zipped file collected between DATEX and DATEY
          \item Contains Tweets of the form
          \item Explain the different terms
          \item Show good clean example and bad example
          \item For comparison using standard source text which is made up of AP Newswiretexts and some Project Gutenburg
          \item Has around X unique words which represents a good size of language
        \end{itemize}
	\item Good Tweet - 20091110083240 user1443@user1444 What a great weekend for football, Cats beat KU and Cowboys beat the Eagles! web
        \item Bad Tweet - 20091110083241 user1458 @user1459 pois é né? tem qe me passar pow, kasopakspoasas.. condideração 0 da sua parte, nao passa nem o Nº p/ abiguinha :p maalaa sz '.' web
        \item Error - 20091110083322 user1801 jz came back home,almost had an accident jz now.... *haih* i hate freaking drivers who dont utilize thier signal lights!!!! argh!!!! =?
\end{itemize}

\section{Evaluation Metric}
\begin{itemize}
	\item Gather sets of 100 Tweets from dataset
	\item Manually read through and correct if needed
	\item Run same Tweets through the system/systems
	\item Compare the system-corrected Tweets with the manually-corrected Tweets
	\item Metric is then the percentage equality between system-corrected and manually-corrected
\end{itemize}

\section{Comparing the Accuracy on Generic Spelling Error Correction}

\subsection{Methods of Experimentation}
\begin{itemize}
	\item Gather different texts that may contain spelling errors
	\item Compare each of the systems including the voting system and an Aspell run on non-Tweets
	\item Repeat 3 times and take average accuracy
	\item Use percentage accuracy to decide which is better
	\item Does any one of the systems/voting system perform better?
\end{itemize}

\subsection{Results and Analysis}
Lots of numbers, explanations and graphs

\section{Comparing the Accuracy on Tweets}
\subsection{Methods of Experimentation}
\begin{itemize}
	\item Collect sets of Tweets and matching corrected Tweets
	\item Compare each of the systems including the voting system and an Aspell run on non-Tweets
	\item Average accuracy over each of the Tweet sets
	\item Use evaluation metric to decide which is better
	\item Does any one of the systems/voting system perform better?
\end{itemize}

\subsection{Results and Analysis}
Lots of numbers, explanations and graphs

\section{Comparing Language Model with Accuracy of Correction}
\subsection{Methods of Experimentation}
\begin{itemize}
	\item Collect sets of Tweets and matching corrected Tweets
	\item Collect 3 different sources of training texts (Modern, Gutenburg, 3rd source?)
	\item Train up system on each of the sources
	\item Compare each of the systems including the voting system and an Aspell run on non-Tweets
	\item Average accuracy over each of the Tweet sets
	\item Use evaluation metric to decide which is better
	\item Does any one of the systems/voting system perform better?
\end{itemize}

\subsection{Results and Analysis}
Lots of numbers, explanations and graphs

\section{Is it Possible to Train on Tweets?}
\subsection{Methods of Experimentation}
\begin{itemize}
	\item Collect training sets of Tweets
	\item Collect sets of Tweets and matching corrected Tweets
	\item Train up system on each of the training sets of Tweets
	\item Compare each of the systems including the voting system and an Aspell run on non-Tweets
	\item Average accuracy over each of the Tweet sets
	\item Use evaluation metric to decide which is better
	\item Does any one of the systems/voting system perform better?
\end{itemize}

\subsection{Results and Analysis}
Lots of numbers, explanations and graphs

\section{Is it possible to Fit Corrections to 140 Characters?}
\subsection{Methods of Experimentation}
\begin{itemize}
	\item Explore each of the results length to see if it's required first
	\item Then explore if anything can be done to shorten Tweets back to 140 characters
\end{itemize}

\subsection{Results and Analysis}
Lots of numbers, explanations and graphs
